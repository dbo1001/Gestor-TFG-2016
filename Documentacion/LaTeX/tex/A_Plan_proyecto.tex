\apendice{Plan de Proyecto Software}

\section{Introducción}
En esta sección se detallará la planificación que se ha realizado, el estudio de viabilidad tanto de la parte económica como de la legal.

\section{Planificación temporal}
\subsection{Sprint 0 (26/10/2020 - 25/11/2020)}
Puesta a punto del proyecto, planteamiento de las herramientas con las que trabajar, búsqueda de alternativas y toma de contacto con las herramientas nuevas que se van a emplear.
Las tareas que se realizaron fueron:
\begin{itemize}
	\tightlist
	\item Añadir la extensión ZenHub al navegador
		Se añadió la extensión Zenhub al navegador de Google Chrome para utilizarlo en el GitHub.
	\item Clonar el repositorio en local. 
		Mediante la aplicación GitHub Desktop se clono el repositorio del Gestor de TFGs mediante el enlace HTTP que proporciona GitHub.
	\item Investigar sobre Vaadin.
		A través de la página oficial de Vaadin se realizo la instalación y me informe acerca de Vaadin.
	\item Actualización del README.md 
		Se modifico el README.md del proyecto para que reflejará los cambios respecto a la versión de la que se hizo fork. 
	\item Investigar LaTeX
		Se procedió a buscar información sobre cómo instalar y manejar Latex para realizar la documentación del proyecto posteriormente.
\end{itemize}

\subsection{Sprint 1 (25/11/2020 - 12/01/2021)}
Generación de test unitarios, búsqueda de trabajos similares, cambio del driver para conectarse con el excel, información para obtener ideas de como realizar ciertas mejoras y comienzo de la documentación del proyecto. Mejora de la cobertura de la aplicación web.

Las tareas planteadas fueron:
\begin{itemize}
	\tightlist
	\item Instalación Miktex + TexStudio
		Tras la reunión del 25/11/20 se decicio que la mejor opción sería emplear un editor local, en vez del editor online Overleaf. Para ello se instalo Miktex junto al editor de texto TexStudio.
	\item Se comienza la documentación en LaTeX  - Spring 0
		Creación de la documentación en LaTeX a partir de las plantillas.
	\item Generar nuevos test
		Para aumentar la covertura de la aplicación se definieron nuevos test.
	\item Cambiar driver JDBC 
		Para poder usar los .ods se buscó una alternativa al driver para .csv que se empleaba.
\end{itemize}

\subsection{Sprint 2 (12/01/2021 - 26/01/2021)}
Búsqueda de nuevos driver que implementen JDBC para ficheros .xls. Prueba de opciones encontradas como Apache POI, SQLSheet, Fillo, Cdata JDBC for Excel y JdbcOdbcDriver. Integrar la API Fillo en el proyecto. 

Las tareas planteadas fueron:
\begin{itemize}
	\tightlist
	\item Instalación del LibreOffice 
		Se instaló Apache LibreOffice para manipular los fichero .xls y .csv que se emplean para la parte de datos de la aplicación.
	\item Cambiar de formato al fichero ``\textbf{BaseDeDatosTFG.ods}''.
		Se modificó el formato del fichero ``\textbf{BaseDeDatosTFG.ods}'' a ``\textbf{BaseDeDatosTFG.xls}''.
	\item Anexos - Modificación para poder cambiar el tamaño de las imágenes.
		Cambios en la plantilla de \textbf{anexos.tex} para poder modificar el tamaño de las imágenes al deseado.
	\item Probar el Apache POI
		Para verificar el funcionamiento de Apache POI, se incluyo en el proyecto de prueba \textbf{HolaMundoVaadin} y se realizaron varios ejemplos de prueba tomando como referente el proyecto principal.
	\item Error al intentar ejecutar la imagen del proyecto gabrielstan/gestion-tfg
		Bug realizado al intentar desplegar el proyecto  gabrielstan/gestion-tfg.  
	\item Desplegar proyectos relacionados con la gestión de TFG
		Proceso por el cual se probó a desplegar los proyectos relacionados con \textbf{GII 20.09 Herramienta web repositorios de TFGII}. 
	\item Memoria - Mejoras
		Se realizaron varios cambios en la memoria.
	\item Incorporación del driver para fichero Excel
		Prueba de la API Fillo en el proyecto de prueba \textbf{HolaMundoVaadin}, y tras verificar su funcionamiento se procedió a incluirlo en el proyecto principal.
	\item Modificación de los nombres de los fichero csv 
		Se cambiaron los nombres de los ficheros .csv para que coincidieran con los nombres de las hojas del fichero .xls. 
	\item Memoria - Documentación Spring 2
		Comienzo de la documentación de lo realizado en el Spring 2. En el cual se detallan las tareas realizadas.
	\item Modificaciones en los test
		Para probar la incorporación de la API Fillo se modificaron el test  ``\textbf{\textbf{SintInfDataTest}}''.
	\item Realización de cambios para el funcionamiento de la nueva conexión para los fichero .xls 
		Para poder usar la API Fillo se han tenido que realizar multitud de cambios en el proyecto. Aunque tanto el ``\textbf{\textbf{CsvDriver}}'' y ``\textbf{\textbf{Fillo}}'' emplean lenguaje SQL, ``\textbf{CsvDriver}'' emplea JDBC puro mientras que ``\textbf{Fillo}'' usa funciones y clases propias.
\end{itemize}

\section{Estudio de viabilidad}
\subsection{Viabilidad económica}
En este apartado se detallan los costes que llevaría realizar este proyecto.

\subsubsection{Coste del personal}

\subsubsection{Coste hardware}
Referente a los costes del equipo utilizado en el desarrollo del trabajo. Teniendo en cuenta el precio del ordenador empleado de aproximadamente 700 euros.

\subsubsection{Coste software}
Referente a los costes de las herramientas software no gratuitas empleadas en el proyecto. Como es el caso del Sistema Operativo Windows o el Microsoft Office 365.

\subsection{Viabilidad legal}
En este apartado se detallaran las licencias de cada dependencia que se ha utilizado en el proyecto

\tablaSmallSinColores{Dependencias del proyecto}{ l | l | l }{dependencias}
{\textbf{Software} & \textbf{Licencia} \\}{
	Vaadin & Apache License 2.0 \\
	Vaadin Maven Plugin & Apache License 2.0 \\
}