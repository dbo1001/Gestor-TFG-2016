\capitulo{3}{Conceptos teóricos}

Se explicaran algunos términos importantes para la comprensión del proyecto.

\section{\textit{Framework}}
Un \textbf{\textit{framework}}~\cite{framework_definicion}, también denominado marco de trabajo, es una estructura conceptual y tecnológico estandarizada que facilita el desarrollo de aplicaciones web mediante un soporte basado en programas,bibliotecas, lenguajes, herramientas, prácticas, criterios, etc.

\section{\textit{Frontend}}
El \textbf{\textit{frontend}}~\cite{frontend_backend} es la parte de una aplicación web o programa que interactúa de forma directa con el usuario. Algunos de los lenguajes más usado para el desarrollo de frontend son: JavaScript, HTML y CSS.

\section{\textit{Backend}}
El \textbf{\textit{backend}}~\cite{frontend_backend} es la parte que se relaciona con la capa de datos, es decir, la base de datos y el servidor. Los usuarios no pueden acceder a ella debido a que contiene lógica de la aplicación. Para el desarrollo del backend se emplean lenguajes como Java y Python.

\section{Desarrollo \textit{full stack}}
El desarrollo \textbf{\textit{full stack}}~\cite{desarrollo_full_stack} es la unión del desarrollo del lado del cliente, \textit{frontend}, y el lado del servidor, \textit{backend}.